\documentclass[12pt,reqno]{amsart}

\usepackage{amsthm,amsmath,amssymb}
\usepackage{mathtools}
\usepackage{geometry}
\usepackage{proof}
\usepackage{xcolor}
\usepackage{graphicx}
\usepackage[T1]{fontenc}
\usepackage{courier}
\usepackage{hyperref}
\hypersetup{
    hidelinks=true
}
\usepackage{listings}
\lstset{basicstyle=\ttfamily\small, columns=fullflexible, language=Lisp, morekeywords={define, lambda, if, car, cdr, zero, eopl, proc, letrec, in, then, else}, keywordstyle=\bfseries\color{blue!40!black}}
\newcommand{\code}[1]{\texttt{#1}}
\graphicspath{ {./} }
\geometry{
 a4paper,
 total={170mm,257mm},
 left=25mm,
 top=20mm,
 }


\begin{document}
\thispagestyle{empty}
\begin{center}
\large\textbf{Problem Set 10 \\ COMP301 Fall 2019} \\
\normalsize\textbf{19.12.2019 17:30 - 18:45} \\
\end{center}

\vspace{7.5mm}

\textbf{Problem 1}: You are given the \code{Letrec} with continuations implemented using \textbf{data-structural representation}. Take your time to understand and digest how it works. You task is to convert this to \code{Letrec} with continuations implemented using \textbf{procedural representation}. You can refer to the lecture slide for help too.

\vspace{7.5mm}

\textbf{Problem 2}: Implement factorial function with continuation passing. You can use the boilerplate given in the codes file.

\vfill

\textbf{This is the last PS! Thank you for coming all this way...}
\end{document}