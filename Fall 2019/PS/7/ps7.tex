\documentclass[12pt,reqno]{amsart}

\usepackage{amsthm,amsmath,amssymb}
\usepackage{mathtools}
\usepackage{geometry}
\usepackage{proof}
\usepackage{xcolor}
\usepackage{graphicx}
\usepackage[T1]{fontenc}
\usepackage{courier}
\usepackage{hyperref}
\hypersetup{
    hidelinks=true
}
\usepackage{listings}
\lstset{basicstyle=\ttfamily\small, columns=fullflexible, language=Lisp, morekeywords={define, begin, end, lambda, if, car, cdr, zero, eopl, proc, letrec, in, then, else}, keywordstyle=\bfseries\color{blue!40!black}}
\newcommand{\code}[1]{\texttt{#1}}
\graphicspath{ {./} }
\geometry{
 a4paper,
 total={170mm,257mm},
 left=25mm,
 top=20mm,
 }


\begin{document}
\thispagestyle{empty}
\begin{center}
\large\textbf{Problem Set 7 \\ COMP301 Fall 2019} \\
\normalsize\textbf{28.11.2019 17:30 - 18:45} \\
\end{center}

\vspace{7.5mm}

\subsection*{Read me first!} Please download the \textit{Codes} file. In the scheme codes, you will see some hints regarding where to modify. You will use DrRacket. We have also edited the \code{tests.rkt} so that if you solve the problem, running \code{tests.rkt} should have no errors. You have only one language to modify in this PS, and both questions are very similar in terms of modification.

\vspace{7.5mm}

\textbf{Problem 1}\footnote{EOPL p.128-129, Exercise 4.29}: Add arrays to \code{mutable-pairs} language. Introduce new operators \code{newarray}, \code{arrayref}, and \code{arrayset} that create, dereference, and update arrays. This leads to:
$$
ArrVal = (Ref(ExpVal))^*
$$
$$
ExpVal = Int + Bool + Proc + ArrVal
$$
$$
DenVal = Ref(ExpVal)
$$
Since the locations in an array are consecutive, use a representation like the second representation above. What should be the result of the following program?
\begin{lstlisting}
let a = newarray(2, -99)
    p = proc (x)
        let v = arrayref(x,1)
        in arrayset(x,1,-(v,-1))
in begin 
    arrayset(a,1,0); 
    (p a); 
    (p a); 
    arrayref(a,1) end
\end{lstlisting}
Here \code{newarray(2,-99)} is intended to build an array of size 2, with each location in the array containing -99. \code{begin} expressions are defined already for you (see exercise 4.4 for them). Make the array indexing zero-based, so for example an array of size 2 should have indices 0 and 1.

\vspace{7.5mm}

\textbf{Problem 2}\footnote{EOPL p.130, Exercise 4.30}: Add to the language of exercise 4.29 (previous problem) a procedure \code{arraylength}, which returns the size of an array. Your procedure should work in constant time. Make sure that \code{arrayref} and \code{arrayset} checks that their indices are within the length of the array.
\end{document}