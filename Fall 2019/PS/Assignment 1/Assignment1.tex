\documentclass[12pt,reqno]{amsart}

\usepackage{amsthm,amsmath,amssymb}
\usepackage{mathtools}
\usepackage{geometry}
\usepackage{proof}
\usepackage{xcolor}
\usepackage{graphicx}
\usepackage[T1]{fontenc}
\usepackage{courier}
\usepackage{hyperref}
\hypersetup{
    hidelinks=true
}
\usepackage{listings}
\lstset{basicstyle=\ttfamily\small, columns=fullflexible, language=Lisp, morekeywords={define, begin, end, lambda, if, car, cdr, zero, eopl, proc, letrec, in, then, else}, keywordstyle=\bfseries\color{blue!40!black}}
\newcommand{\code}[1]{\texttt{#1}}
\graphicspath{ {./} }
\geometry{
 a4paper,
 total={170mm,257mm},
 left=25mm,
 top=20mm,
 }


\begin{document}
\thispagestyle{empty}
\begin{center}
\large\textbf{Assignment 1 \\ COMP301 Fall 2019} \\
\normalsize\textbf{Due 05.12.2019 23:59} \\
\end{center}


\textbf{Task:} Explain the differences between dynamic and static (lexical) scoping. What are their advantages and disadvantages? You should consider the benefits not just in terms of writing the code, but debugging it too. Also, look at the program below:
\begin{lstlisting}
let a = 3
in let p = proc (x) -(x, a)
    in let a = 5
        in -(a, (p 2)) 
\end{lstlisting}
% Static evaluates to 6, dynamic evalutes to 8.
In your answer, include an explanation on how the program is evaluated and what the result is, both for static and dynamic scoping.

\textbf{Submission:} Write you answer to the and submit a word or pdf file. Do not exceed 500 words. 

\end{document}